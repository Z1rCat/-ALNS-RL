\documentclass[11pt]{article}
\usepackage[margin=1in]{geometry}
\usepackage{graphicx}
\usepackage{amsmath,amssymb}
\usepackage{booktabs}
\usepackage{hyperref}
\usepackage{ctex}
\usepackage{float}
\setlength{\parskip}{0.6em}
\setlength{\parindent}{0pt}

\title{分布漂移可视化与课程式跳级机制的实现\\
\large ALNS--RL 组会汇报稿(偏汇报,轻论文)}
\author{}
\date{}

\begin{document}
\maketitle

\section{任务背景:把“分布变换”显式展示在 RL Reward 曲线中}

本工作的系统背景是 \textbf{ALNS--RL 协同的多式联运动态优化框架}。

\begin{itemize}
\item \textbf{ALNS}:负责仿真与路径/调度优化。每一个环境实例对应一个独立的 Excel 文件;
\item \textbf{RL(DQN / A2C / PPO)}:在运行过程中,当出现不确定事件(如拥堵、延误)时,做高层离散决策;
\item \textbf{动作空间}:
\begin{itemize}
\item $a=0$:等待 / 保持当前方案(Keep / Wait);
\item $a=1$:触发重规划(Reroute)。
\end{itemize}
\item \textbf{智能跳级(Curriculum Jumping)}:当 RL 达到阶段性收敛条件后,不必按 $0\ldots499$ 的顺序跑完整个环境序列,而是根据场景结构跳转到下一个阶段或直接进入测试;
\item \textbf{物理隔离(\texttt{run\_dir})}:每一次运行都输出到独立目录
\texttt{codes/logs/run\_*/},包含 \texttt{data/}、日志 CSV 与元配置文件,用于复现实验与对比分析。
\end{itemize}

在上一次组会上,老师提出了一个关键意见:

\begin{quote}
\emph{在最终绘制 RL reward 曲线时,需要把环境的分布切换也明确展示出来,否则 reward 曲线缺乏环境背景。}
\end{quote}

这一意见直接推动了我后续的一系列实现与反思。本次汇报的核心工作,正是围绕
\textbf{“分布漂移如何被正确展示、以及这暴露了哪些原有机制的问题”}展开。

\section{实现可视化后暴露的问题:RL 可能“过早收敛”}

在将分布切换显式画入 reward 曲线之后(Fig.~2 右轴 + 背景分段),我发现了一个此前被忽略的重要漏洞。

原有的智能跳级逻辑是:
\begin{itemize}
\item 只要 RL 的 rolling average reward 达到阈值,就认为当前阶段已经学会;
\item 一旦满足该条件,就允许直接进入测试阶段。
\end{itemize}

这一逻辑在 \textbf{单阶段或 OOD 设计} 中是合理的,
但在 \textbf{多阶段场景}(如 A$\to$B 或 A$\to$B$\to$A)下会产生严重偏差:

\begin{itemize}
\item RL 可能在阶段 A 很快达标;
\item 但尚未充分经历阶段 B(甚至完全没进入 B);
\item 就直接切换到测试阶段,导致“适应性”“记忆能力”等结论不再严谨。
\end{itemize}

因此,我将“跳级”从 \textbf{仅由收敛触发},
升级为 \textbf{由场景结构与收敛条件共同决定}。

\section{关键改动一:为环境实例打标签并写入日志}

\subsection{训练与测试的实例划分}

每一次实验统一生成 500 个环境实例
(\texttt{table\_number = 0\ldots499})。

\begin{itemize}
\item 训练阶段:前 350 个实例(概念上对应 $0\ldots349$);
\item 测试阶段:后 150 个实例(概念上对应 $350\ldots499$)。
\end{itemize}

系统内部实现中,测试阶段采用 \textbf{倒序运行}(从 499 递减到 350)。

在近期调试中,我发现当 \texttt{implement=1} 切换发生时,
如果 ALNS 线程继续自增/自减 \texttt{table\_number} 但并未真正运行实例,
会出现“看似进入测试、但立刻触及边界”的假象。

为此,我补充了 \textbf{implement 切换时的索引同步与等待机制},
确保测试阶段确实从 499 开始运行
(对应 \texttt{Dynamic\_ALNS\_RL34959.py} 中的
\texttt{implement\_start\_synced} 逻辑)。

\subsection{环境标签的生成与传递}

每一个 Excel 环境文件都包含一个 \texttt{\_\_meta\_\_} sheet,至少记录:

\begin{itemize}
\item \texttt{gt\_mean}:该实例所属分布的均值;
\item \texttt{phase\_label}:阶段标签(如 A / B / C)。
\end{itemize}

RL 在构建状态时读取该 sheet,并将对应信息写入日志。

\section{关键改动二:基于时序结构特性的六类分布设计与能力假设}

为了全面评估 RL 在动态环境下的鲁棒性与适应性,我们基于 \texttt{distribution\_config.json} 构建了六类典型的分布模式。

本设计的核心创新在于:\textbf{将 500 个环境实例的时序生成逻辑(Pattern)与 $0\ldots349$(训练集)/ $350\ldots499$(测试集)的切分点进行精确对齐}。通过控制分布切换点相对于测试切分点的位置,我们为每一类场景赋予了明确的“能力探针”语义。

\subsection{分布类型定义与验证目标}

表 \ref{tab:distribution_design} 展示了六类分布的具体结构及其对应的核心假设。所有分布生成均基于 $N=500$ 的总长度,其中索引 $350$ 为训练与测试的硬性物理隔离界限。

\begin{table}[h]
\centering
\caption{六类分布的时序结构设计与核心验证目标}
\label{tab:distribution_design}
\renewcommand{\arraystretch}{1.3} % 增加行高
\resizebox{\textwidth}{!}{% 自适应宽度
\begin{tabular}{lp{2cm}p{5cm}p{6cm}}
\toprule
\textbf{类别} & \textbf{Pattern} & \textbf{时序结构 (Indices $0 \to 499$)} & \textbf{核心验证目标 (Hypothesis)} \\
\midrule
\textbf{基准} (S1) & \texttt{random\_mix} & \textbf{全程混合} \newline A/B 以 50/50 概率随机交替 & \textbf{IID 鲁棒性}:验证 RL 能否在无时序规律的噪声中,学到一种“平均最优”策略,而非过拟合某一次特定的波动。 \\
\hline
\textbf{泛化} (S3) & \texttt{ab} & \textbf{分布阶跃} \newline A: $0\ldots349$ (Train) \newline B: $350\ldots499$ (Test) & \textbf{OOD 零样本泛化}:训练阶段从未见过分布 B。验证当环境发生结构性突变时,RL 是否具备基本的通用性(Generalization),还是会因过拟合 A 而彻底失效。 \\
\hline
\textbf{适应} (S5) & \texttt{adaptation} & \textbf{早期突变陷阱} \newline A: $0\ldots99$ \newline B: $100\ldots499$ (覆盖测试段) & \textbf{克服过早收敛}:A 阶段极短,RL 易在此处误判为收敛并停止。此场景专门验证“跳级机制”能否识别环境变化,强制 RL 继续学习 B 分布。 \\
\hline
\textbf{记忆} (S2) & \texttt{aba} & \textbf{回归结构} \newline A($0\ldots174$) $\to$ B($175\ldots349$) $\to$ \textbf{A($350\ldots499$)} & \textbf{灾难性遗忘}:测试集回归到初始分布 A。验证 RL 在为了适应中间的 B 阶段而调整参数后,能否“回想起”旧策略,还是发生了灾难性遗忘? \\
\hline
\textbf{演变} (S6) & \texttt{abc} & \textbf{连续漂移} \newline A $\to$ B $\to$ \textbf{C($350\ldots499$)} & \textbf{持续学习能力}:训练经历 A$\to$B,测试进入全新的 C。模拟现实中不断恶化的复杂非平稳环境。 \\
\hline
\textbf{风险} (V1) & \texttt{aba} (均值恒定) & \textbf{方差突变} \newline $\mu_A = \mu_B$,但 $\sigma_A \neq \sigma_B$ & \textbf{风险敏感性}:当期望收益(均值)不变时,验证 RL 是风险厌恶(偏向 Wait)还是风险寻求(偏向 Reroute)。 \\
\bottomrule
\end{tabular}%
}
\end{table}

\subsection{关键设计逻辑解析}

\begin{enumerate}
    \item \textbf{OOD 的绝对隔离 (S3\_1 \texttt{ab}):} 
    我们将 A 分布严格限制在 $0\ldots349$,这意味着在训练阶段,RL \textbf{绝对无法}接触到 B 分布的数据。测试阶段的性能完全取决于算法对状态空间的\textit{外推能力},而非插值记忆。这是最严苛的测试场景。
    
    \item \textbf{适应性的陷阱设计 (S5\_1 \texttt{adaptation}):}
    设计 $A(0\ldots99) \to B$ 的初衷是制造一个“收敛陷阱”。传统的 Early Stopping 往往会在 Index $50\ldots80$ 左右因为 Reward 趋稳而触发停止,从而完全错过后续的 B 分布。该模式强迫系统必须引入\textbf{基于环境特征的跳级检查},否则训练出的模型在测试集(全为 B)将表现极差。
    
    \item \textbf{记忆与遗忘的博弈 (S2\_1 \texttt{aba}):}
    测试集被刻意安排为回归 A 分布($350\ldots499$)。这要求 RL 模型不仅要适应中间的 B,还必须在神经网络权重中保留 A 的特征映射。如果模型为了适应 B 而“覆写”了 A 的知识,测试得分将显著低于基准策略。
\end{enumerate}
\section{关键改动三:按分布类型定制跳级逻辑}

\subsection{RL 侧的阶段性收敛判定}

RL 侧维护一个用于跳级的阶段性收敛信号(\texttt{curriculum\_converged}):

\begin{itemize}
\item 滑动窗口长度:30;
\item 收敛阈值:0.7(Debug 模式下降为 0.3);
\item 连续达标次数:3 次。
\end{itemize}

需要强调的是:该信号\textbf{并不代表最终最优收敛},而仅用于控制算力在不同阶段之间的分配。

\subsection{ALNS--RL 协同的跳级控制}

跳级控制逻辑位于 \texttt{Dynamic\_ALNS\_RL34959.py},
由 ALNS 线程维护当前 \texttt{table\_number}。

一旦 RL 达到阶段性收敛,就根据 \texttt{distribution\_config.json} 中定义的 \texttt{pattern} 决定下一步:

\begin{itemize}
\item 多阶段场景:跳转到下一阶段的起点;
\item OOD 场景(S3):训练只跑 A,达标后直接进入测试(B);
\item 记忆 / 风险场景:训练覆盖所有关键阶段,再在测试中验证回归行为。
\end{itemize}

\section{典型场景的跳级示例}

\subsection{S5\_1:适应场景(A$\to$B)}

目标是避免 RL 在 A 阶段快速学会后直接进入测试,从而使“适应性”论证失效。

具体逻辑为:

\begin{enumerate}
\item 在 A 阶段达标后,不进入测试;
\item 将 \texttt{table\_number} 跳转到 B 的起点;
\item 在 B 中满足最小训练量后,才允许进入测试;
\item 测试阶段从 499 开始倒序运行。
\end{enumerate}

\subsection{V1\_3:风险场景(均值恒定,方差突变)}

由于均值不变,单独绘制 \texttt{gt\_mean} 无法体现环境切换。

因此在可视化中采用自动指标切换:

\begin{itemize}
\item 若检测到方差变化且均值恒定,则右轴绘制 \texttt{gt\_std};
\item 图内显式标注“均值恒定为 $X$”。
\end{itemize}

\section{机制创新:基准策略回放与反事实对比验证体系}

在强化学习(RL)模型的评估中,单纯依赖平均奖励(Average Reward)往往难以全面衡量智能体的真实能力。例如,在低干扰环境(如对数正态分布均值为 3 的场景)中,简单的“始终等待(Always-Wait)”策略即可获得理论最优解。若 RL 仅获得高分,无法证明其具备了状态感知与复杂决策能力。

因此,为了量化 RL 相对于启发式规则的“智能溢价(Intelligence Premium)”,本研究构建了包含三种基准策略(Always-Wait, Always-Reroute, Random)的对比体系。同时,针对本框架特有的“智能跳级(Curriculum Jumping)”机制导致的训练序列非线性问题,设计并实现了**基准策略回放(Benchmark Replay)**机制,以确保在完全一致的上下文环境中进行公平对比。

\subsection{基准策略回放机制:轨迹对齐与环境复用}

由于本研究引入了基于收敛阈值的跳级机制,RL Agent 在训练和测试阶段经历的算例序列(\texttt{table\_number})是动态变化的(例如,模型可能在 Index 100 快速收敛后直接跳跃至 Index 350)。传统的静态测试集评估法无法复现 RL 做出决策时的时序上下文。

为此,本研究提出了一种**轨迹对齐(Trajectory Alignment)**的回放机制,通过脚本 \texttt{run\_benchmark\_replay.py} 实现反事实评估(Counterfactual Evaluation):

\begin{enumerate}
    \item \textbf{环境强复用(Environment Persistence)}:回放过程直接读取 RL 运行时生成的沙盒目录(\texttt{run\_dir/data/})下的 Excel 环境文件,严格禁止重新生成不确定性事件。这确保了基准策略与 RL 面对的是完全相同的“试卷”,消除了随机种子带来的环境噪音。
    \item \textbf{序列重放(Sequence Replay)}:系统解析 RL 的运行日志 \texttt{rl\_trace.csv},提取出 RL 实际经历的 $(table\_number, phase)$ 精确时序序列。基准策略将严格按照此顺序执行,包括相同的跳级点和倒序测试路径。
    \item \textbf{闭环同步(Closed-loop Synchronization)}:基准脚本并非简单地离线计算,而是像 RL Agent 一样与 ALNS 仿真器进行实时交互。它根据历史 Phase 设置仿真参数,发出固定的基准动作,并轮询等待 ALNS 返回真实的 Reward。这保证了底层仿真器的状态转移与 RL 运行时完全一致。
\end{enumerate}

\subsection{三级基准策略体系}

为了全方位评估 RL 的智能程度,我们定义了三种不同层级的基准策略 $\pi_{base}$:

\begin{itemize}
    \item \textbf{保守基准:总是选择 0 (Always-Wait, $\pi_{wait}$)}:
    \begin{equation}
        a_t = 0, \quad \forall s_t
    \end{equation}
    始终选择“等待/保持原计划”。该策略在低均值、低方差的简单环境中往往是最优解。引入此基准旨在验证 RL 是否学会了“懒惰”——即在不必要重规划时,能否像保守策略一样节省计算成本和操作成本,避免过度反应。
    
    \item \textbf{激进基准:总是选择 1 (Always-Reroute, $\pi_{reroute}$)}:
    \begin{equation}
        a_t = 1, \quad \forall s_t
    \end{equation}
    始终触发“重规划”。该策略在高均值、高风险的复杂环境中通常能通过绕路减少延误损失。引入此基准旨在验证 RL 在面对严重干扰时,能否逼近甚至超越激进策略的收益上限,体现对风险的响应能力。
    
    \item \textbf{噪声基准:随机策略 (Random, $\pi_{rand}$)}:
    \begin{equation}
        a_t \sim \text{Bernoulli}(0.5)
    \end{equation}
    以 50\% 的概率随机选择动作。用于排除环境自身的偶然性(例如某些死局下无论如何选择收益均相同),确立性能下界(Lower Bound),证明 RL 的优势来源于有效的特征学习而非运气。
\end{itemize}

\subsection{深度对比目标与累计优势指标}

通过将 RL 的性能曲线与上述基准策略进行同坐标系对比(参见实验结果部分的 Fig 4 等),本研究旨在回答以下三个核心问题:

\begin{enumerate}
    \item \textbf{有效性(Effectiveness)}:RL 是否在绝大多数时间步中击败了 Random 和表现较差的那个确定性基准?
    \item \textbf{状态敏感性(State Sensitivity)}:RL 是否展现出了动态切换能力?即在低风险区逼近 Always-Wait 的表现,而在高风险区逼近 Always-Reroute 的表现。
    \item \textbf{分布适应性(Distribution Adaptation)}:当环境发生分布漂移(如 OOD 场景)导致某一固定基准策略系统性崩溃时,RL 是否能保持鲁棒性?
\end{enumerate}

为了直观量化 RL 相对于基准策略的持续性收益,我们定义了**“累计优势(Cumulative Advantage)”**指标 $\mathcal{V}_t$。该指标计算 RL 累积获得的奖励与基准策略累积奖励的差值,其定义如下:

\begin{equation}
    \mathcal{V}_t(\pi_{base}) = \sum_{k=0}^{t} \left( r_k(\pi_{RL}) - r_k(\pi_{base}) \right)
\end{equation}

其中,$r_k(\pi)$ 表示策略 $\pi$ 在第 $k$ 个决策步(对应同一环境文件)获得的单步奖励。
\begin{itemize}
    \item 当曲线斜率为正($\nearrow$)时,表明 RL 在该阶段持续优于基准;
    \item 当曲线走平($\rightarrow$)时,表明 RL 表现与基准持平(通常意味着 RL 收敛到了该环境下的最优基准策略);
    \item 当曲线斜率为负($\searrow$)时,表明 RL 表现劣于基准(通常发生在探索初期)。
\end{itemize}
这一指标能够清晰地展示 RL 在环境突变前后的损益变化过程。
\section{多维可视化验证体系}

为了从机制、现象到量化结果全方位验证 ALNS-RL 框架的有效性,本研究升级了可视化脚本 \texttt{plot\_paper\_figure.py}。该系统针对每个实验 Run 生成四张核心图表,构成了完整的证据链。

以下以 S3\_1 场景(OOD 泛化测试:训练于常态分布,测试于未见过的瘫痪分布)为例进行说明:

\subsection{图 1:环境分布漂移全景图 (Environment Shift)}
\begin{figure}[H]
    \centering
    \includegraphics[width=0.95\textwidth]{A:/MYpython/34959_RL/ALNS_Research_Documentation/latex/paper_figures/run_S3_1_A2C/fig1_environment.pdf}
    \caption{环境生成机制验证:时序漂移(左)与统计分布特征(右)}
    \label{fig:env_shift}
\end{figure}

\textbf{解读指南:}
\begin{itemize}
    \item \textbf{左子图(时序散点)}:展示了算例序列(Table Number)对应的环境压力(Severity/Mean)。蓝色点群代表训练阶段(低压常态),红色点群代表测试阶段(高压瘫痪)。虚线清晰标记了分布发生结构性突变(Shift)的时刻。
    \item \textbf{右子图(核密度估计 KDE)}:展示了环境参数的概率分布。双峰(或明显分离的单峰)形态证明了训练集与测试集存在显著的 \textbf{Covariate Shift},确立了 OOD(Out-of-Distribution)实验的物理基础。
\end{itemize}

\subsection{图 2:适应性 S 曲线 (Adaptation Dynamics)}
\begin{figure}[H]
    \centering
    \includegraphics[width=0.95\textwidth]{A:/MYpython/34959_RL/ALNS_Research_Documentation/latex/paper_figures/run_S3_1_A2C/fig2_adaptation.pdf}
    \caption{RL 适应性轨迹与基准策略对比}
    \label{fig:adaptation_curve}
\end{figure}

\textbf{解读指南:}
\begin{itemize}
    \item \textbf{双轴设计}:左轴为平滑奖励(Reward),右轴(黑色点状阶梯线)为环境真实均值(GT Mean)。
    \item \textbf{核心看点}:
    \begin{enumerate}
        \item \textbf{突变响应}:当右轴黑线阶梯式上升(环境变难)时,观察蓝色 RL 曲线是否出现短暂下跌(Shock);
        \item \textbf{恢复斜率}:观察 RL 曲线是否能快速回升并收敛,形成 S 型适应曲线;
        \item \textbf{基准压制}:对比 RL 曲线是否显著高于灰色虚线(Always-Wait)和橙色虚线(Always-Reroute)。在 S3\_1 中,我们期望看到 RL 在测试阶段(灰色背景区)依然保持高位,而 Wait 策略应大幅崩盘。
    \end{enumerate}
\end{itemize}

\subsection{图 3:细粒度策略热力图 (Granular Policy Shift)}
\begin{figure}[H]
    \centering
    \includegraphics[width=0.95\textwidth]{A:/MYpython/34959_RL/ALNS_Research_Documentation/latex/paper_figures/run_S3_1_A2C/fig3_policy_heatmap.pdf}
    \caption{策略行为模式演变分析}
    \label{fig:policy_heatmap}
\end{figure}

\textbf{解读指南:}
\begin{itemize}
    \item \textbf{图表含义}:展示了 RL Agent 在不同阶段(Y轴:Train vs Implement)对四种细分动作(X轴)的选择频率。
    \item \textbf{机制证明}:这是解释 RL \textbf{“学到了什么”} 的关键。
    \item \textbf{预期现象}:在 S3\_1 中,训练阶段(低压)颜色应集中在 \texttt{Removal-Wait}(倾向保守);而进入测试阶段(高压)后,高亮区域应显著转移至 \texttt{Removal-Reroute}(倾向激进)。这种 \textbf{策略漂移 (Policy Shift)} 证明了模型具备状态感知能力,而非盲目记忆。
\end{itemize}

\subsection{图 4:累计优势趋势图 (Cumulative Advantage)}
\begin{figure}[H]
    \centering
    \includegraphics[width=0.95\textwidth]{A:/MYpython/34959_RL/ALNS_Research_Documentation/latex/paper_figures/run_S3_1_A2C/fig4_cumulative_advantage.pdf}
    \caption{RL 相对于基准策略的累计净收益}
    \label{fig:cumulative_advantage}
\end{figure}

\textbf{解读指南:}
\begin{itemize}
    \item \textbf{指标定义}:$\mathcal{V}_t = \sum (R_{RL} - R_{Baseline})$。
    \item \textbf{趋势判断}:
    \begin{itemize}
        \item \textbf{向上倾斜(正斜率)}:RL 持续优于基准策略,差距不断扩大(碾压优势)。
        \item \textbf{水平直线}:RL 表现与基准持平(通常发生在最优解显而易见的简单场景)。
        \item \textbf{向下倾斜}:RL 表现劣于基准(通常仅在训练初期的探索阶段出现)。
    \end{itemize}
    \item \textbf{价值证明}:如果在测试阶段(曲线后半段)呈现持续上升趋势,即证明了 RL 算法在应对不确定性时具有显著的工程价值与鲁棒性。
\end{itemize}
\section{工程实现:批量实验编排与并行计算架构}

面对“多分布模式 $\times$ 多算法类型 $\times$ 多请求规模($R$) $\times$ 多基准策略”带来的组合爆炸式计算需求,本研究构建了一套分层级、高鲁棒的自动化实验编排系统。该系统旨在实现从本地开发环境(MVP)到服务器高性能计算环境(HPC)的无缝迁移。

\subsection{分层调度体系}
为平衡调试灵活性与大规模计算效率,我们实现了三级调度脚本:
\begin{itemize}
    \item \textbf{公共调度内核 (\texttt{run\_experiments\_common.py})}:封装了实验配置解析、任务队列管理与异常捕获逻辑,作为核心库被上层调用。
    \item \textbf{本地调试入口 (\texttt{run\_experiments\_local.py})}:面向开发环境优化,默认采用小规模样本与低并发度,用于快速验证代码逻辑与数据流闭环(即“最小可跑样例”)。
    \item \textbf{服务器全量入口 (\texttt{run\_experiments\_server.py})}:面向生产环境优化,支持全量参数矩阵的一键部署,并集成了自动重试与断点续跑机制。
\end{itemize}

\subsection{原子任务全链路闭环}
系统将单次实验定义为一个不可分割的“原子任务单元(Atomic Task Unit)”,每个单元严格执行以下标准化闭环流程:
\begin{enumerate}
    \item \textbf{主程序执行}:调用 \texttt{Dynamic\_master34959.py},完成从环境生成、RL 模型训练到测试阶段的全过程,并将所有运行数据落盘至独立的沙盒目录(Sandbox Directory)。
    \item \textbf{反事实评估}:自动触发 \texttt{run\_benchmark\_replay.py --policy all},在相同的环境序列下回放 Wait/Reroute/Random 等基准策略,生成对比数据。
    \item \textbf{自动化绘图}:调用 \texttt{plot\_paper\_figure.py},基于生成的日志自动产出符合期刊标准的四张核心图表,实现“无人值守”的证据生成。
\end{enumerate}

\subsection{并发策略与 IO 安全机制}
为了在最大化利用算力的同时保证数据完整性,本系统设计了特定的并发控制策略:

\begin{itemize}
    \item \textbf{沙盒隔离并发}:以“独立 Run ID 目录”为最小并发单元。不同实验进程在物理文件系统上完全隔离,彻底消除了写冲突(Write Conflicts)风险。
    \item \textbf{资源感知调度}:基于 \texttt{ThreadPoolExecutor} 构建进程池。默认并发数动态设定为 $N_{cpu} - 2$,既充分压榨算力,又预留核心用于系统 IO 与调度开销,防止系统假死。
    \item \textbf{IO 安全生成模式}:针对 Excel 文件并发写入极易导致损坏(如 \texttt{BadZipFile/EOFError})的工程痛点,系统在“不确定性事件生成”阶段强制启用\textbf{单核模式}(\texttt{generator\_workers=1})。虽然牺牲了微小的生成速度,但保证了二进制文件的绝对安全,实现了计算效率与数据稳定性的最优平衡。
\end{itemize}


\section{结果展示与分析(六组实验,24 张图)}

本节对六次代表性运行的实验结果进行完整展示与分析。每一次运行均包含四类图像:
环境结构(Fig1)、reward 适应曲线(Fig2)、动作策略分布(Fig3)以及相对基准策略的累计优势(Fig4)。
所有图像均来自对应 run 的 \texttt{paper\_figures/} 子目录。

为避免混淆,下文每一小节均明确给出 \textbf{分布类型 + 算法 + run ID},
并且所有 “Fig1–Fig4” 均指该 run 目录下的同名文件:
\texttt{fig1\_environment.pdf},
\texttt{fig2\_adaptation.pdf},
\texttt{fig3\_policy\_heatmap.pdf},
\texttt{fig4\_cumulative\_advantage.pdf}。

% ============================================================
\subsection{V1\_3:均值恒定、方差突变(风险敏感性)}
% ============================================================

\subsubsection{V1\_3 + DQN\\
(run\_20260120\_123246\_371223\_R30\_V1\_3\_DQN\_S42)}

\begin{figure}[H]\centering
\includegraphics[width=0.9\textwidth]{paper_figures/run_V1_3_DQN/fig1_environment.pdf}
\caption{V1\_3 场景的环境结构:均值恒定,方差在中段突变}
\end{figure}

Fig1 显示该场景在整个过程中环境均值保持不变,仅在中段引入显著更高的方差。
这意味着从理论上看,高层最优决策策略在不同阶段并不存在结构性变化,
该场景主要用于检验 RL 对不确定性“方差变化”的风险敏感性。

\begin{figure}[H]\centering
\includegraphics[width=0.9\textwidth]{paper_figures/run_V1_3_DQN/fig2_adaptation.pdf}
\caption{DQN 在 V1\_3 场景下的 reward 适应曲线(右轴为环境方差)}
\end{figure}

Fig2 表明,在方差显著增大的阶段,DQN 的 reward 并未出现系统性下降,
整体曲线保持平稳,说明该场景下不存在明显的适应失败问题。

\begin{figure}[H]\centering
\includegraphics[width=0.9\textwidth]{paper_figures/run_V1_3_DQN/fig3_policy_heatmap.pdf}
\caption{DQN 在 V1\_3 场景下的动作频率与条件平均奖励}
\end{figure}

Fig3 显示,实施阶段策略几乎完全塌缩为“等待”动作。
结合条件平均奖励可以看到,在该场景下等待本身就是收益最高的决策,
因此这种策略塌缩是\textbf{合理且正确的最优行为},而非策略失效。

\begin{figure}[H]\centering
\includegraphics[width=0.9\textwidth]{paper_figures/run_V1_3_DQN/fig4_cumulative_advantage.pdf}
\caption{DQN 在 V1\_3 场景下相对基准策略的累计优势}
\end{figure}

Fig4 中,相对 Always-Wait 的累计优势接近 0,
而相对 Always-Reroute 的优势持续累积,
进一步验证了该策略在风险场景下的合理性。

% ------------------------------------------------------------

\subsubsection{V1\_3 + A2C\\
(run\_20260120\_175128\_674609\_R30\_V1\_3\_A2C\_S42)}

\begin{figure}[H]\centering
\includegraphics[width=0.9\textwidth]{paper_figures/run_V1_3_A2C/fig1_environment.pdf}
\caption{V1\_3 场景的环境结构(A2C)}
\end{figure}

\begin{figure}[H]\centering
\includegraphics[width=0.9\textwidth]{paper_figures/run_V1_3_A2C/fig2_adaptation.pdf}
\caption{A2C 在 V1\_3 场景下的 reward 适应曲线}
\end{figure}

与 DQN 类似,A2C 在该场景下的 reward 表现同样稳定,
说明两种算法在“均值不变”的环境中都不会出现系统性失效。

\begin{figure}[H]\centering
\includegraphics[width=0.9\textwidth]{paper_figures/run_V1_3_A2C/fig3_policy_heatmap.pdf}
\caption{A2C 在 V1\_3 场景下的动作策略分布}
\end{figure}

不同之处在于,Fig3 显示 A2C 并未将重规划动作完全压制,
而是保留了少量高收益但低频的探索行为,
体现出相较 DQN 更为细腻的风险—收益权衡。

\begin{figure}[H]\centering
\includegraphics[width=0.9\textwidth]{paper_figures/run_V1_3_A2C/fig4_cumulative_advantage.pdf}
\caption{A2C 在 V1\_3 场景下的累计优势}
\end{figure}

% ============================================================
\subsection{S3\_1:中等均值漂移(OOD 泛化)}
% ============================================================

\subsubsection{S3\_1 + DQN\\
(run\_20260120\_113326\_895986\_R30\_S3\_1\_DQN\_S42)}

\begin{figure}[H]\centering
\includegraphics[width=0.9\textwidth]{paper_figures/run_S3_1_DQN/fig1_environment.pdf}
\caption{S3\_1 场景的环境结构:训练 A,测试 B}
\end{figure}

Fig1 显示该场景在训练阶段仅暴露低均值分布 A,
而测试阶段切换到更高均值的分布 B,
这是一个典型的分布外(OOD)泛化测试。

\begin{figure}[H]\centering
\includegraphics[width=0.9\textwidth]{paper_figures/run_S3_1_DQN/fig2_adaptation.pdf}
\caption{DQN 在 S3\_1 场景下的 reward 变化}
\end{figure}

Fig2 表明在分布切换发生后,DQN 的 reward 明显下降,
且在后续过程中长期停留在中低水平,缺乏明显的二次适应迹象。

\begin{figure}[H]\centering
\includegraphics[width=0.9\textwidth]{paper_figures/run_S3_1_DQN/fig3_policy_heatmap.pdf}
\caption{DQN 在 S3\_1 场景下的动作策略分布}
\end{figure}

Fig3 显示 DQN 的策略高度偏向等待,
尽管在条件平均奖励意义下重规划具有更高收益,
但该动作被系统性抑制,体现出明显的保守性。

\begin{figure}[H]\centering
\includegraphics[width=0.9\textwidth]{paper_figures/run_S3_1_DQN/fig4_cumulative_advantage.pdf}
\caption{DQN 在 S3\_1 场景下的累计优势}
\end{figure}

累计优势结果显示,DQN 虽整体优于弱基准(Always-Wait),
但相对强基准(Always-Reroute)的优势有限,
反映其在 OOD 场景下的泛化能力不足。

% ------------------------------------------------------------

\subsubsection{S3\_1 + A2C\\
(run\_20260120\_113326\_897492\_R30\_S3\_1\_A2C\_S42)}

\begin{figure}[H]\centering
\includegraphics[width=0.9\textwidth]{paper_figures/run_S3_1_A2C/fig1_environment.pdf}
\caption{S3\_1 场景的环境结构(A2C)}
\end{figure}

\begin{figure}[H]\centering
\includegraphics[width=0.9\textwidth]{paper_figures/run_S3_1_A2C/fig2_adaptation.pdf}
\caption{A2C 在 S3\_1 场景下的 reward 适应曲线}
\end{figure}

与 DQN 相比,A2C 在分布切换后虽经历初期下降,
但随后 reward 出现明显回升并趋于稳定,
表明其具备一定的分布外再适应能力。

\begin{figure}[H]\centering
\includegraphics[width=0.9\textwidth]{paper_figures/run_S3_1_A2C/fig3_policy_heatmap.pdf}
\caption{A2C 在 S3\_1 场景下的动作策略分布}
\end{figure}

Fig3 显示 A2C 的策略并未完全塌缩,
等待与重规划在实施阶段的条件平均奖励接近,
说明其能够在新分布下重新组织策略结构。

\begin{figure}[H]\centering
\includegraphics[width=0.9\textwidth]{paper_figures/run_S3_1_A2C/fig4_cumulative_advantage.pdf}
\caption{A2C 在 S3\_1 场景下的累计优势}
\end{figure}

% ============================================================
\subsection{S5\_1:强均值漂移(快速适应极限)}
% ============================================================

\subsubsection{S5\_1 + DQN\\
(run\_20260119\_233229\_651769\_R30\_S5\_1\_DQN\_S42)}

\begin{figure}[H]\centering
\includegraphics[width=0.9\textwidth]{paper_figures/run_S5_1_DQN/fig1_environment.pdf}
\caption{S5\_1 场景的环境结构:强均值漂移}
\end{figure}

\begin{figure}[H]\centering
\includegraphics[width=0.9\textwidth]{paper_figures/run_S5_1_DQN/fig2_adaptation.pdf}
\caption{DQN 在 S5\_1 场景下的 reward 变化}
\end{figure}

Fig2 显示在强分布切换发生后,DQN 的 reward 明显下探,
且几乎不存在后续恢复过程。

\begin{figure}[H]\centering
\includegraphics[width=0.9\textwidth]{paper_figures/run_S5_1_DQN/fig3_policy_heatmap.pdf}
\caption{DQN 在 S5\_1 场景下的动作策略分布}
\end{figure}

Fig3 表明 DQN 的策略几乎完全塌缩为等待,
即便重规划在条件平均奖励上显著更优,
仍然很少被选择,体现出明显的过度保守性。

\begin{figure}[H]\centering
\includegraphics[width=0.9\textwidth]{paper_figures/run_S5_1_DQN/fig4_cumulative_advantage.pdf}
\caption{DQN 在 S5\_1 场景下的累计优势}
\end{figure}

% ------------------------------------------------------------

% 【修复1】使用 [] 作为 PDF 书签标题,避免 \\ 报错
\subsubsection[S5\_1 + A2C]{S5\_1 + A2C \\ (run\_20260119\_233229\_651769\_R30\_S5\_1\_A2C\_S42)}

\begin{figure}[H]\centering
\includegraphics[width=0.9\textwidth]{paper_figures/run_S5_1_A2C/fig1_environment.pdf}
\caption{S5\_1 场景的环境结构(A2C)}
\end{figure}

\begin{figure}[H]\centering
\includegraphics[width=0.9\textwidth]{paper_figures/run_S5_1_A2C/fig2_adaptation.pdf}
\caption{A2C 在 S5\_1 场景下的 reward 变化}
\end{figure}

A2C 在该场景下同样经历性能下降,
但 reward 并未完全失控,表现出一定的鲁棒性。

\begin{figure}[H]\centering
\includegraphics[width=0.9\textwidth]{paper_figures/run_S5_1_A2C/fig3_policy_heatmap.pdf}
\caption{A2C 在 S5\_1 场景下的动作策略分布}
\end{figure}

Fig3 显示 A2C 在高压阶段仍保留一定的动作结构,
能够在部分情形下重新利用重规划或插入动作。

\begin{figure}[H]\centering
\includegraphics[width=0.9\textwidth]{paper_figures/run_S5_1_A2C/fig4_cumulative_advantage.pdf}
\caption{A2C 在 S5\_1 场景下的累计优势}
\end{figure}

\section{阶段性实验结论与深度洞察}

通过对 DQN 与 A2C 两种典型强化学习算法在不同分布漂移场景(S3\_1, S5\_1, V1\_3)下的系统性对比,结合基准策略回放机制生成的累计优势曲线,本研究得出以下具有重要方法论意义的结论:

\subsection{1. 策略塌缩的双重性:从“最优收敛”到“灾难性过拟合”}
实验通过对比 V1\_3(方差突变)与 S3\_1/S5\_1(均值突变)揭示了“策略塌缩(Policy Collapse)”现象的本质差异:

\begin{itemize}
    \item \textbf{良性塌缩(Optimal Convergence)}:在 V1\_3 场景(均值恒定为 15,仅方差增大)中,DQN 的策略完全塌缩为“等待(Wait)”(Fig 7),A2C 也高度倾向于等待(Fig 11)。结合累计优势图(Fig 8, Fig 12)可见,此时 RL 表现与 Always-Wait 基准持平且优于 Always-Reroute。这表明:\textbf{在均值未发生结构性改变的场景下,单一动作往往就是全局最优解}。此时的策略塌缩是算法成功收敛到最优启发式规则的体现,而非失效。
    \item \textbf{恶性塌缩(Catastrophic Overfitting)}:在 S3\_1 和 S5\_1 场景(均值从 9 突变至 90)中,DQN 依然塌缩在“等待”动作上(Fig 15, Fig 23),导致 Reward 断崖式下跌且无法恢复(Fig 14, Fig 22)。与之形成鲜明对比的是,Always-Reroute 基准在测试段表现优异(Fig 16)。这证明 DQN \textbf{过拟合了训练阶段的低压环境特征},丧失了探索新最优策略的能力。这种塌缩在分布发生结构性漂移(Structural Shift)时是致命的。
\end{itemize}

\subsection{2. 算法适应性差异机理:DQN 的刚性 vs. A2C 的柔性}
在应对强均值漂移(S3\_1, S5\_1)时,A2C 展现出了显著优于 DQN 的鲁棒性(Fig 18 vs Fig 14):

\begin{itemize}
    \item \textbf{DQN (Value-based)}:由于依赖 $\epsilon$-greedy 策略进行探索,在训练后期 $\epsilon$ 衰减后,DQN 极易陷入局部极小值。当环境从“无需重规划”突变为“必须重规划”时,DQN 仍基于旧有的 Q 值表过高估计“等待”的价值,导致其无法感知新环境下的动作优势反转。
    \item \textbf{A2C (Actor-Critic)}:得益于其显式的策略建模(Policy Gradient)和熵正则化(Entropy Regularization),A2C 在测试阶段仍保留了非零的动作概率分布(Fig 19, Fig 27 展示了 A2C 在高压区保留了约 15\%-20\% 的 Reroute 动作)。这种\textbf{策略的“柔性”}使其在环境突变后能够快速试错并捕捉到重规划带来的高额回报,从而实现了 S 型适应曲线的爬升。
\end{itemize}

\subsection{3. 环境漂移类型的敏感度分级}
实验表明 ALNS-RL 框架对不同统计特征变化的敏感度存在显著层级:

\begin{itemize}
    \item \textbf{均值漂移(Mean Shift)是核心挑战}:环境均值的数量级变化(如 9 $\to$ 90)直接改变了最优策略的性质(从 Wait $\to$ Reroute)。这是导致模型泛化失败的主因,也是验证“适应性”的最有效试金石。
    \item \textbf{方差漂移(Variance Shift)主要影响风险偏好}:在 V1\_3 中,方差的剧增并未导致模型性能崩溃。这说明在期望值不变的情况下,当前的 RL 算法能够保持鲁棒,并未因环境噪声的增加而迷失方向。
\end{itemize}

\subsection{4. 基准回放体系的验证价值}
引入 \texttt{baseline\_wait} 和 \texttt{baseline\_reroute} 的反事实对比,为归因分析提供了决定性证据:
\begin{itemize}
    \item 如果没有 Always-Reroute 作为参照(绿色曲线),我们可能误以为 S5\_1 测试段的低分是由于环境本身“无解”造成的。
    \item 通过 Fig 16(DQN vs Baselines),我们清晰地看到 RL 跑输了 Always-Reroute,从而确凿地判定这是\textbf{算法适应性失败},而非环境不可行。这一机制成功排除了环境噪音干扰,准确定位了算法缺陷。
\end{itemize}

\section{后续计划与讨论问题}

下一步计划包括:

\begin{itemize}
\item Docker 化部署,固定依赖与绘图环境;
\item 服务器全量实验:多分布 $\times$ 多算法 $\times$ 多随机种子;
\item 汇总 Mean$\pm$Std 结果并形成对比表格。
\end{itemize}

% 【修复2】删除了这里多余的 \end{itemize}

\end{document}
