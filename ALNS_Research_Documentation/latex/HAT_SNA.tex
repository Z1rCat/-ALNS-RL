\documentclass[11pt]{article}
\usepackage[margin=1in]{geometry}
\usepackage{amsmath,amssymb}
\usepackage{booktabs}
\usepackage{graphicx}
\usepackage{hyperref}
\usepackage{float}
\usepackage{ctex}
\usepackage{algorithm}
\usepackage{algpseudocode}

\setlength{\parskip}{0.6em}
\setlength{\parindent}{0pt}

\title{基于历史注意力的非平稳单步决策改进:PPO/A2C 的 HAT--SNA 方法与工程实现}
\author{项目:ALNS+RL 双线程系统(34959)}
\date{\today}

\begin{document}
\maketitle

\section{汇报目的与结论}
本文件用于向老师汇报:我们在现有 ALNS+RL 双线程系统中,\textbf{在不改动 SB3(PPO/A2C) 核心优化器的前提下},通过\textbf{历史包装器(History) + 注意力特征提取(Attention)}增强策略对非平稳分布突变的适应性,并给出实现逻辑、公式与复现实验信息。

\textbf{命名:}HAT = History-Attention Transform;SNA = Simple Nonstationarity Adaptation(漂移感知调度 + 推理期门控;红阶段严格无梯度更新)。

\textbf{最关键改动点:}将原始低维观测 $o_t$ 扩展为长度为 $H$ 的历史序列 $X_t$,并用轻量 Transformer 编码为上下文特征 $f_t$,再交给 SB3 的 actor/critic 网络;同时在\textbf{单步 episode} 场景下通过 \textbf{keep\_history} 机制让历史跨 reset 保留,使模型在 episode\_length=1 时仍能形成有效的 $h_t$。

\textbf{一个直接证据:}以同一场景(R30, S5, seed=42)为例,A2C 基线在测试阶段 200 次决策的平均回报约为 0.505,而 A2C\_HAT\_SNA 达到约 0.805(详见 \S\ref{sec:results})。

\section{问题设置:单步决策 + 隐含非平稳模式}
\subsection{系统交互(ALNS+RL 双线程)}
系统由 ALNS 线程生成决策点状态并阻塞式等待 RL 决策;RL 线程输出动作后由 ALNS 计算并回填奖励。每个决策为单步 episode:
\begin{itemize}
  \item 动作空间:$a_t \in \{0,1\}$(例如 \texttt{wait/keep} vs \texttt{reroute})。
  \item 奖励:$r_t \in \{0,1\}$(二值成功/失败反馈)。
  \item 观测:$o_t \in \mathbb{R}^d$(低维,包含如 \texttt{delay\_tolerance}、\texttt{severity} 等;严禁使用显式 regime 标签作为输入)。
\end{itemize}

\subsection{非平稳性:隐藏模式分段稳定并突变}
环境存在隐藏模式 $z_t$(分段稳定、突变),RL 只能通过历史 $h_t$ 推断:
\begin{align}
z_t &\sim \text{piecewise-constant process}, \\
o_t &\sim p(o \mid z_t), \qquad r_t \sim p(r \mid o_t, a_t, z_t).
\end{align}
由于 episode\_length=1,若每次 reset 都清空历史,则策略退化为纯 contextual bandit:$\pi(a\mid o_t)$ 无法利用跨时信息识别 $z_t$,从而对突变适应极慢。

\section{基线:SB3 的 PPO / A2C(保持不改)}
\subsection{Actor--Critic 基本形式}
我们复用 stable-baselines3 的 PPO/A2C 实现。对任意时刻 $t$,策略与价值函数为:
\begin{align}
\pi_\theta(a_t \mid s_t) \quad\text{与}\quad V_\phi(s_t),
\end{align}
其中 $s_t$ 为输入特征(基线为 $s_t=o_t$,本文改进为 $s_t=f_t$)。

\subsection{单步 episode 下的优势}
在单步 episode(done 每步为真)条件下,回报目标自然为:
\begin{align}
G_t = r_t, \qquad A_t \approx r_t - V_\phi(s_t).
\end{align}
SB3 内部仍使用通用的 GAE/TD 形式,但由于 done 信号每步截断,实际训练信号接近上述单步优势。

\subsection{A2C 的损失(SB3 原生,本文不修改)}
以单步优势为例,A2C 的典型目标可写为最小化:
\begin{align}
\mathcal{L}_{\text{A2C}}(\theta,\phi)
&=
\underbrace{-\mathbb{E}_t\bigl[\log \pi_\theta(a_t \mid s_t)\,A_t\bigr]}_{\text{policy loss}}
+ c_V \underbrace{\mathbb{E}_t\bigl[(V_\phi(s_t)-G_t)^2\bigr]}_{\text{value loss}}
- c_H \underbrace{\mathbb{E}_t\bigl[\mathcal{H}(\pi_\theta(\cdot\mid s_t))\bigr]}_{\text{entropy bonus}}.
\end{align}
其中 $\mathcal{H}(\cdot)$ 为熵,$c_V,c_H$ 为系数。本文的 HAT 只改变 $s_t$ 的构造方式(从 $o_t$ 变为 $f_t$),不改 A2C 的优化目标与更新流程。

\subsection{PPO 的 clipped 目标(SB3 原生,本文不修改)}
PPO 的核心为 clipped surrogate objective。令重要性比率
\begin{align}
\rho_t(\theta)=\frac{\pi_\theta(a_t\mid s_t)}{\pi_{\theta_{\text{old}}}(a_t\mid s_t)},
\end{align}
则其策略目标可写为:
\begin{align}
\mathcal{L}^{\text{clip}}_{\text{PPO}}(\theta)
=
\mathbb{E}_t\left[
\min\left(
\rho_t(\theta)A_t,\;
\text{clip}(\rho_t(\theta),1-\epsilon,1+\epsilon)A_t
\right)
\right],
\end{align}
并配合 value loss 与 entropy bonus 组成总目标。本文的 HAT 同样只改变 $s_t$(输入特征),不改 PPO 的核心裁剪机制。

\section{HAT:History + Attention 的模型层改进}
\label{sec:hat}
\subsection{动机:让策略看到 $h_t$ 而非只看 $o_t$}
我们将任务视为 POMDP:单步 episode 仅是执行机制,真实信息在跨步历史中。因此我们构造固定窗口历史:
\begin{align}
h_t = \bigl(o_{t-H+1}, a_{t-H}, r_{t-H}, \ldots, o_t \bigr),
\end{align}
并学习一个上下文编码器 $f_t = g_\psi(h_t)$,让 actor/critic 都在 $f_t$ 上工作。

\subsection{HistoryAttentionWrapper:历史序列化(无未来泄漏)}
我们定义动作 one-hot 编码:
\begin{align}
e(a) \in \{0,1\}^2,\quad e(a)[a]=1.
\end{align}
构造 token(只使用\textbf{过去}的动作与回报,不使用未来信息):
\begin{align}
x_t = \bigl[\,o_t;\; e(a_{t-1});\; r_{t-1}\,\bigr] \in \mathbb{R}^{d+2+1}.
\end{align}
将最近 $H$ 个 token 堆叠得到输入序列(不足 $H$ 时零填充):
\begin{align}
X_t = [x_{t-H+1}, \ldots, x_t] \in \mathbb{R}^{H\times (d+3)}.
\end{align}
\textbf{数据泄漏约束:}任何显式的分布真值标签(例如 phase\_label/regime\_id 等)\textbf{仅用于日志与评估},不进入 $o_t$/$x_t$/$X_t$。
\textbf{关键工程点:}由于 episode\_length=1,环境会频繁调用 reset。为避免历史被清空,我们引入 \texttt{keep\_history} 开关:当其为真时,reset 不清空历史队列,从而跨 episode 维护 $X_t$。

\begin{algorithm}[H]
\caption{HistoryAttentionWrapper(核心逻辑)}
\begin{algorithmic}[1]
\State 初始化:队列 $\mathcal{D}\leftarrow [\;]$(最大长度 $H$),$\texttt{last\_a}\leftarrow \mathbf{0}$,$\texttt{last\_r}\leftarrow 0$
\Function{Reset}{}
\State $o \leftarrow \texttt{env.reset()}$
\If{\texttt{keep\_history} 为假 或 $\mathcal{D}$ 为空}
\State 清空 $\mathcal{D}$,重置 $\texttt{last\_a},\texttt{last\_r}$
\EndIf
\State 追加 $x=[o;\texttt{last\_a};\texttt{last\_r}]$ 到 $\mathcal{D}$
\State \Return $\texttt{stack\_and\_pad}(\mathcal{D}) \in \mathbb{R}^{H\times(d+3)}$
\EndFunction
\Function{Step}{$a$}
\State $(o', r, done, info) \leftarrow \texttt{env.step}(a)$
\State $\texttt{last\_a}\leftarrow e(a)$,$\texttt{last\_r}\leftarrow r$
\State 追加 $x'=[o';\texttt{last\_a};\texttt{last\_r}]$ 到 $\mathcal{D}$
\State \Return $(\texttt{stack\_and\_pad}(\mathcal{D}), r, done, info)$
\EndFunction
\end{algorithmic}
\end{algorithm}

\subsection{AttentionExtractor:轻量 Transformer 编码历史}
给定 $X_t \in \mathbb{R}^{H\times D}$($D=d+3$),先线性嵌入并加位置编码:
\begin{align}
u_i &= W_e x_i + b_e,\quad p_i \in \mathbb{R}^{m}, \\
v_i &= u_i + p_i,\qquad i=1,\ldots,H.
\end{align}
对每一层 Transformer self-attention(以单头为例):
\begin{align}
Q &= V W_Q,\quad K = V W_K,\quad V' = V W_V,\\
\mathrm{Attn}(Q,K,V') &= \mathrm{softmax}\left(\frac{QK^\top}{\sqrt{m}}\right)V'.
\end{align}
多头注意力与前馈网络堆叠 $L$ 层后,取最后一个 token 的表示作为 pooled 上下文:
\begin{align}
c_t &= \mathrm{TransformerEncoder}(X_t)_{H},\\
f_t &= W_p c_t + b_p \in \mathbb{R}^{F}.
\end{align}
其中 $F$ 为输出特征维度(我们实现中默认 $F=64$),作为 SB3 policy 的特征输入。

\textbf{要点:}attention 的作用不是追求“复杂”,而是提供一种\textbf{可学习的历史聚合}:模型可以在非平稳突变后,依靠历史 token 的统计变化来重构隐模式 $z_t$ 的“证据”,从而改变策略输出。

\section{SNA:非平稳增强(训练期调参 + 推理期门控)}
\label{sec:sna}
\subsection{漂移统计(EMA)}
为了在不改变主优化器的情况下增强非平稳适应,我们引入运行态统计量(不是网络参数):
\begin{align}
\bar r_t &= (1-\alpha)\bar r_{t-1} + \alpha r_t,\\
d_t &= |r_t - \bar r_t|,\\
\bar d_t &= (1-\alpha)\bar d_{t-1} + \alpha d_t,
\end{align}
其中 $\alpha\in(0,1)$ 为 EMA 系数。

\subsection{训练阶段:漂移感知的超参缩放(不改梯度公式)}
当检测到更大的 $\bar d_t$ 时,我们对 SB3 算法的部分超参做温和放缩(\textbf{仅训练阶段}生效):
\begin{itemize}
  \item PPO:增大 $\texttt{clip\_range}$ 与 $\texttt{target\_kl}$(更允许策略变化,缩短适应延迟);
  \item A2C:增大 $\texttt{ent\_coef}$(增加探索,避免在旧分布下过早“锁死”动作偏好)。
\end{itemize}
放缩形式可以写为:
\begin{align}
\eta_t = \min(\eta_{\max}, 1 + s \cdot \bar d_t), \qquad \text{hyperparam}_t = \text{hyperparam}_0 \cdot \eta_t,
\end{align}
其中 $s$ 为尺度系数,$\eta_{\max}$ 限制最大放大倍率。

\subsection{实施阶段(红阶段):严格冻结,只有推理期自适应}
系统硬约束要求红阶段\textbf{禁止任何梯度/优化器更新}。因此我们仅允许\textbf{纯前向}机制改变最终动作。
我们使用一个简单的门控阈值规则(不改变网络参数):
\begin{align}
g_t &= \mathbb{I}[\bar d_t > \tau_d \;\;\text{or}\;\; \bar r_t < \tau_r],\\
\tau(g_t) &= \begin{cases}
\tau_{\text{low}}, & g_t=1 \\
\tau_{\text{high}}, & g_t=0
\end{cases},\\
a_t &= \mathbb{I}[p_\theta(a=1\mid X_t) \ge \tau(g_t)].
\end{align}
其中 $p_\theta(a=1\mid X_t)$ 来自冻结策略网络的输出概率。该机制属于 inference-time adaptation:\textbf{不更新参数,但允许基于历史统计改变动作判别阈值}。

\textbf{说明:}如果实施阶段系统侧不提供逐步 reward 回传到 RL 线程,则 $\bar r_t,\bar d_t$ 只能在训练/评估阶段更新(门控在红阶段退化为“使用最后一次统计”)。这属于系统接口层面限制,与 HAT 的模型能力无冲突。

\section{工程实现:文件、开关与复现实验}
\subsection{关键代码组件(对应本仓库)}
\begin{itemize}
  \item \textbf{历史 + 注意力模块:}\texttt{codes/robust\_rl/sb3\_attention.py}
    \begin{itemize}
      \item \texttt{HistoryAttentionWrapper}: 将观测改为 $H\times(d+3)$ 序列,并支持 \texttt{keep\_history}。
      \item \texttt{AttentionExtractor}: TransformerEncoder + last-token pooling,输出 $F$ 维特征给 SB3。
    \end{itemize}
  \item \textbf{训练/实施主逻辑与集成:}\texttt{codes/dynamic\_RL34959.py}
    \begin{itemize}
      \item 当环境变量 \texttt{RL\_HAT=1} 且算法为 PPO/A2C 时,自动包裹 env 并注入 \texttt{policy\_kwargs}。
      \item 训练阶段调用 \texttt{model.learn(...)}(SB3 原生),只改变输入特征与少量超参调度(SNA)。
      \item 实施阶段保持 \textbf{no online learning};动作选择可使用阈值门控版本(SNA 的 inference 部分)。
    \end{itemize}
  \item \textbf{运行入口与算法名映射:}\texttt{codes/Dynamic\_master34959.py}
    \begin{itemize}
      \item 支持 \texttt{--algorithm PPO\_HAT} / \texttt{A2C\_HAT}(以及对应 run 命名),内部设置 \texttt{RL\_HAT=1} 并选择 SB3 基算法。
    \end{itemize}
\end{itemize}

\subsection{关键超参(默认值)}
HAT(模型层):
\begin{itemize}
  \item 历史长度:$H=20$(\texttt{HATConfig.history\_len})
  \item Transformer:层数 $L=2$,heads=2,embed\_dim=64,dropout=0.1
  \item 输出特征维:$F=64$(\texttt{HATConfig.feature\_dim})
  \item keep\_history:默认开启(适配 episode\_length=1)
\end{itemize}
SNA(运行态统计/调度):
\begin{itemize}
  \item EMA 系数:$\alpha=0.05$(\texttt{HAT\_EMA\_ALPHA})
  \item 门控阈值:$\tau_d=0.4$,$\tau_r=0.3$(\texttt{HAT\_GATE\_DRIFT\_HI}, \texttt{HAT\_GATE\_REWARD\_LOW})
  \item 推理期阈值:$\tau_{\text{high}}=0.6$,$\tau_{\text{low}}=0.4$(\texttt{HAT\_TAU\_HIGH}, \texttt{HAT\_TAU\_LOW})
  \item 调度倍率:$s=1.5$,$\eta_{\max}=3.0$(\texttt{HAT\_DRIFT\_SCALE}, \texttt{HAT\_DRIFT\_MAX\_SCALE})
\end{itemize}

\subsection{复现实验命令(示例)}
\begin{verbatim}
python codes/Dynamic_master34959.py --dist_name S5_1 --request_number 30 --algorithm A2C_HAT --seed 42 --workers 1
python codes/Dynamic_master34959.py --dist_name S5_1 --request_number 30 --algorithm A2C     --seed 42 --workers 1
\end{verbatim}
每次运行会生成独立 run 目录:\texttt{codes/logs/run\_*/},包含 \texttt{rl\_trace.csv}、\texttt{rl\_training.csv}、\texttt{rl\_summary.csv} 等。

\section{结果示例(A2C vs A2C\_HAT\_SNA)}
\label{sec:results}
我们以两次 run 的 \texttt{rl\_summary.csv} 为例(测试阶段 200 次决策统计):
\begin{table}[H]
\centering
\begin{tabular}{lccc}
\toprule
方法 & 平均回报 & 标准差 & 决策数 \\
\midrule
A2C(baseline, S5, R30, seed42) & 0.505 & 0.500 & 200 \\
A2C\_HAT\_SNA(S5, R30, seed42) & 0.805 & 0.396 & 200 \\
\bottomrule
\end{tabular}
\caption{测试阶段平均回报对比(来自对应 run 的 \texttt{rl\_summary.csv})。}
\end{table}

\textbf{解读:}在该场景中,HAT 通过历史序列与注意力聚合形成上下文特征,使策略能依据近期观测/行动/回报模式做更一致的选择,从而显著提高测试阶段平均回报。

\section{边界与已知限制(如需老师进一步指导的点)}
\begin{itemize}
  \item 若红阶段不提供逐步 reward 回传,任何基于 reward 的门控/漂移检测都无法在线更新,只能依赖观测统计或训练期学习出的内部表征。
  \item 当前 HAT 使用固定窗口 $H$ 与 last-token pooling;后续可以探索更强的池化(attention pooling)或多专家头(MoE)以进一步缩短突变后的翻转延迟(仍保持红阶段无梯度更新)。
\end{itemize}

\end{document}
